\documentclass[a4paper, 10pt, oneside]{report}
\usepackage[utf8]{inputenc}
\usepackage[T1]{fontenc}
\usepackage[francais]{babel}
\usepackage[top=2cm, bottom=2cm, left=2cm, right=2cm]{geometry}

\title{Plan d'assurance qualité}
\author{Vincent \bsc{Moittie}, Romain \bsc{Notari} et Christophe \bsc{Sin Ronia}}
\date{18 mai 2014}
\begin{document}
\maketitle        %Page de garde
\tableofcontents  %Sommaire
\chapter{Les besoins et limite « qualité » du projet}

Description du contexte du projet ainsi que d’une qualité limitée par le petit nombre de membres composant le groupe projet

\chapter{Contexte d’utilisation du PAQ}

Par rapport à ce contexte, description de la manière dont va être utilisé et suivi le PAQ tout au long de la réalisation effective du projet. Descriptif des aspects contractuels de ce PAQ par rapport aux activités liées à la finalité (scolaire mais également d’exploitation par les utilisateurs) du projet

\chapter{Domaine d'application}

Identification des entités (personnes, services, systèmes automatisés) internes et/ou externes impliquées ainsi que les grandes lignes d’actions dans chaque phase Remarque : Chaque action « Qualité » devra être planifiée.

\section{Phases de l’avant-projet}

Contrôles et revues qualités prévus pour être réalisés tout au long de l’avant-projet (Période de la planification allant de MT1 à MT4)

\section{Phases du développement}

Contrôles et revues qualités prévus pour être réalisés tout au long du développement (Période de la planification allant de MT5 à MT7 et MT10)

\section{Phases du pilote}

Contrôles et revues qualités prévus pour être réalisés tout au long de la mise en production du ou des pilotes (Période de la planification TE8.1)

\section{Phases du déploiement}

Contrôles et revues qualités prévus pour être réalisés tout au long de la généralisation de la mise en production (Période de la planification TE8.2)

\chapter{Responsabilité de l’assurance qualité}

\section{Responsable qualité}

Nomination d’un membre de l’équipe projet en tant que responsable qualité

\section{Procédures palliatives}

Descriptif des procédures à suivre en cas de non-application du PAQ en distinguant le cas du non-respect sans dérogation avec celui de demande de dérogation.

\section{Gestion des évolutions du PAQ}

Gestion des évolutions fonctionnelles (processus métiers)
Remarque : Pour le projet, puisqu’il est limité en périmètre et qu’il n’y a aucune évolution technologique à prendre en compte il n’est nécessaire de ne gérer que les deltas fonctionnels

\chapter{Suivi de l’exécution du plan}

Chaque action prévisionnelle devra être planifiée.

\section{Actions de contrôles}

Type de revue (doc, logiciel, etc...), contrôles aléatoires (planifiés que sur les plannings des membres « qualité », audit qualité de la phase (analyse de « l’état qualité »)

\section{Les réunions « Qualité » et les participants}

\section{Les procédures de validation des CR}

Les procédures de validation des CR qualité avec la liste des destinataires (ici les 3 référents du projet)


\chapter{Références documentaires}

Liste des documents utilisés pour définir le management
Remarque : Ici il s’agit du support de cours et du template sous ISO10006 mis en ligne

\chapter{Lexique}

Acronymes, abréviations, etc...

\end{document}